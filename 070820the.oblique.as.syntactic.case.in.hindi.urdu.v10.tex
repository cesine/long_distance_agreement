\documentclass{article}
\usepackage{covington}
\usepackage{fullpage}
%\usepackage{tree-dvips}
\usepackage{qtreegina}
\usepackage{xylinggina}
%\usepackage{xyling}
\usepackage{tipa}
\usepackage{ulem}
%\usepackage{natbib} %citation style
\usepackage{graphicx}
\let\ipa\textipa %to use \ipa rather than \textipa

\author{Gina Chiodo}
\title {A Syntactic Analysis of the Oblique form in Hindi/Urdu\thanks{I would like to thank the Abbasi family for sharing their home and language with me. Thanks especially to Karthik Durvasula for extensive discussion and his always willing ear. Thanks also to Masahiro Yamada, Lanny Hidajat, Timothy Mckinnon Emmy Cathcart and Nadya Pincus. All errors and oversights are my own :) The following abbreviations are used in this paper:
%Abs - Absolutive;
ACC - Accusative;
DAT - Dative;
ERG - Ergative;
GEN - Genitive;
Neg - Negation;
Prs - Present;
%LDA - Long Distance Agreement;
Hon - Honorific;
D - Default;
Pstprt - Past Participle;
Nmlz - Nominalizer;
%Pass - Passive Auxiliary;
Psi - Polarity Sensitive Item;
Inf - Infinitive;
Pst - Past;
Pfv - Perfective;
Subj - Subjunctive;
Top - Topic Marker;
Foc - Focus;
Impfv - Imperfective;
Imper -  Imperative;
Prog - Progressive;
Hab - Habitual;
%Ger - Gerund;
OBL - Oblique;
1 - 1st Person;
2 - 2nd Person;
3 - 3rd Person;
F - Feminine;
M - Masculine;
N - Neuter;
Sg - Singular;
Pl - Plural}}
\date{April 2007}
\begin{document}
%\bibliographystyle{linquiry2}
\pagenumbering{arabic}
\maketitle

 \begin{abstract}In this paper I provide a description of the Hindi/Urdu oblique noun forms. To date the oblique has not been mentioned in the Hindi/Urdu case and agreement literature in the Minimalist/P\&P/GB frameworks. I argue that the oblique noun forms are the realization of an abstract structural case. I provide an explicit analysis in DM framework using Legates abstract vs morpholoigcal case, exploded DPs and feature percolatoin.
\end{abstract}

\tableofcontents

\section{Introduction}

In the lineage of Goverment and Binding, Principles and Parameters, Minimalist and Distributed Morphology the oblique noun forms of Hindi/Urdu have yet to be discussed. The oblique form is not resticted to nouns, but rather demonstratives, wh-words, adjectives, nouns and gerunds can appear in the oblique form. In this paper I show that all elements of a DP which is a complement of a postposition appear in the oblique form. I claim that the oblique is the realization of a structural case. The adjective, gerund and noun forms are easily decomposed into a stem and a suffix. Some of the demonstratives and wh-words can also be broken into a stem and a suffix while others are indivisible portmanteus.

The varied overt realization of the oblique can be accounted for if one adopts Legate's division of abstract vs morphological case. Abstract case is determined in the syntax and is realized as morphological case depending on the resources of the langauges lexicon. With out this abstraction away from actual morphological realization of case the distribution of oblique noun forms appears sporadic and mysterious.

The presence of an affixal case (the oblique) which appears on the complements of postpositions raises questions about the status of the Hindi/Urdu postpositions as case assigners. This is particularly relevant for discussions of case and agreement in Hindi/Urdu which has independatly come to the conclusion that the Hindi/Urdu `case postpositions' do not behave croslingusticlallly as expected, Bhatt claims that agreement and case cannot be the realizations of one relation, Bhatt also reports that hindi/urud as an exception ot burzio's generalization. Legate claims that hindi urdu has determined agreement, the verb will agree with the object if the subject is case marked (for her casemarking is a postposition) this falls out naturally if postpostionsarent case but rather funcitonal heads which do not permit the DP to be an argument..



\section{Proposal for Hindi/Urdu Case System}

\begin{example}Proposal:\\
\begin{tabular}{lllllll}
Case Feature & nominative & T$^o$ & \o \\
& accusative & v$^o$ & \o \\
& oblique & P$^o$ & -e,\~{V}\\
Functional Head & dative && =ko\\
 & ergative & & =ne\\
 & genitive && =ka/ki/ke \\
& instrumental, && =se\\
& source && =se \\
& in && =m\~{e}\\
& on && =par \\
& toward, until && =tak \\
& through && dvara\\
\end{tabular}
\end{example}

\section{Previous Proposals for Hindi/Urdu Case System}

\begin{example}Previous discusison of case:\\
\begin{tabular}{lccccccccc}
& Mahajan 1990 & Mohanan 1994 & Butt 2004\\
nom & \o & \o & \o \\
acc & \o & =ko & =ko\\
erg & =ne & =ne & =ne\\
dat & =ko & =ko & =ko\\
instr && =se & =se\\
gen && =ka/ki/ke & =ka/ki/ke\\
loc && m\~{e} & m\~{e}\\
&& par & par\\
&& & tak\\
&& & \o \\
%OBL&& Non-Nom & \\
\end{tabular}
\end{example}

Previous proposals of Hindi/Urdu case only consider postpositions.

\begin{example}
``all case is realized by postpositions, except for nominative case.” (Davison 2004:202)
\end{example}

However, there are also Hindi/Urdu noun forms which alternate between `direct' and `oblique' forms, the `oblique' is found on the complement of a postposition in (2), the `direct' is found elsewhere in (3). The forms are obligatory (4), (5).

\begin{example}Oblique form of gadhe
\glll us gadhe se le lo
[us \ipa{g@d\textsuperscript{H}}-e se] l-e l-o
that.OBL donkey-M.Sg.OBL from take-Subj take-2.SG
\glt `Take it from that donkey.'
\glend
\end{example}

\begin{example}Direct form of gadha\\
\gll hamara gadhaa ghaas~to nahin khaa~taa
1.Pl-GEN-M donkey-M grass=Top Neg eat=Impfv-M
\glt `Our donkey doesn't eat grass.'
\glend
\end{example}

\begin{example}The oblique is obligatory when the NP is the complement of a postposition\\
\gll [*wo   gadha   se]  le lo
that  donkey.MSG  from  take-SUBJ take-2nd.sg
\glt`Take it from that donkey.'
\glend
\end{example}

\begin{example}The oblique cannot appear without a postposition\\
\gll [*us   gadhe]   ghaas  kha rahaa   hai
that.OBL  donkey.MSG.OBL  grass eat IMPERF.M.SG 3.SG
\glt`That donkey is eating grass.'
\glend
\end{example}

The oblique form is similar to the accusative m on `him' in the English structure below.

\begin{example}Complements of prepositions in English are accusative
 I took the grass [ from him.CASE] \\
\Tree {
 & PP\B{dl}\B{dr}\\
from\Below{  [CASE]$\rightarrow$ } && NP\B{d}\\
&&him \\
}
\end{example}

\begin{example}Complements of  postpositions in Urdu are oblique
\gll us \ipa{g@d\textsuperscript{H}}{\bf e} =se le =lo
that donkey-CASE from take take-2nd
\glt ``Take it from that donkey."
\glend
\Tree {
 & PP\B{dl}\B{dr}\\
NP\B{d} && se\Below{ $\leftarrow$ [CASE] }\\
\ipa{gad\textsuperscript{H}}e && \\
}
\end{example}

I claim that the oblique is also a form of case, a suffixal case rather than postpositional.

\section{Description of the ``oblique"}

The exact shape of the oblique stem depends on the final phonological segment, and the gender (masculine and feminine) of the word.(Mohanan 1994:61)

In () below the final -a ending of the `direct' masculine gadha `donkey' alternates with the oblique  -e in gadhe `donkey-OBL.'

\begin{example}The oblique forms of nouns (Shukla 2001)\\
\begin{tabular}{|l|l|l|l|l|l|}\hline
& \multicolumn{2}{c|}{Singular} & \multicolumn{2}{c|}{Plural} &  \\\hline
& Direct & Oblique & Direct & Oblique & \\\hline\hline
~~~~~Masculine  & & & & & \\\hline
a) & gadha & gadhe & gadhe & gadh\~{o} & `donkey'\\\hline
b) & mez & mez & ? & ? & `table'\\\hline\hline
~~~~~Feminine  & & & & &\\\hline
c) & gadhi & gadhi & gadhi\~{a} & gadhi\~{o} & `female donkey'\\\hline
d) & bahu & bahu & bahu\~{e} & bahu\~{o} & `daughter-in-law'\\\hline
\end{tabular}
\end{example}

Pronouns also show a `direct'/`oblique alternation. In () below the nominative stem for 1st singular is mɛ͂ while the oblique stem is mʊjh.

\begin{example}The forms of pronouns (based on Butt \& King 2004:174, and Shukla 2001:188-) \\
Function+Distance+Root+Gender.Oblique.Indefinite+Plural\\
Oblique works for =se,=par,=tak=\ipa{m\~{E}}\\
=ne, =ko and genitive are exceptions for person pronouns, but for different reasons\\
\begin{tabular}{|ll|ll|l|l|l|}\hline
Person & Function & \multicolumn{3}{c|}{Singular} & \multicolumn{2}{c|}{Plural}  \\\hline
&& \multicolumn{2}{c|}{Direct} & Oblique & Direct & Oblique  \\\hline
&& Masculine & Feminine & & & \\\hline\hline
1 && \ipa{m\~{E}} && mujh se & ham & ham se\\
2 & Disrespect & tu && tujh se \\
2 & Familiar& tum && tum se & tum log & tum log-\~{o}\\
2 & Formal & ap && ap se & ab log & ab log-\~{o}\\\hline
3 & Prox & ye && i-s se & ye & i-n(h\~{o}) se\\
3 & Distal & v-o && u-s se & v-o & u-n(h\~{o}) se\\
3 & Rel & j-o && j-i-s & j-o & j-i-n-(h\~{o})\\
Person & Wh  & \ipa{k-On} && k-i-s se & \ipa{k-On} & k-i-n-(h\~{o}) se\\
Person & Indef & k-o-i && k-i-s-i se & k-o-i - ko-i & k-i-s-i - k-i-s-i se\\\hline
Quantity & Prox & i-tn-a & i-tn-i & i-tn-e\\
Quantity & Distal & u-tn-a & u-tn-i & u-tn-e\\
Quantity & Rel & j-i-tn-a & j-i-tn-i & j-i-tn-e \\
Quantity & Wh & k-i-tn-a & k-i-tn-i & k-i-tn-e\\\hline
Manner & Prox & es-a & es-i & es-e\\
Manner & Distal & ve-s-a & ve-s-i & ve-s-e\\
Manner & Rel & je-s-a & je-s-i & je-s-e\\
Manner & Wh & ke-s-a & ke-s-i & ke-s-e\\\hline
Place & Prox & y-ah-\~{a} \\
Place & Distal & v-ah-\~{a}\\
Place & Rel & j-ah-\~{a}\\
Place & Wh & k-ah-\~{a}\\
Place & Indef & k-ah-\~{i}\\\hline
Time & Rel & j-ab\\
Time & Wh & k-ab\\
Time & Indef & k-ab-i\\\hline\hline
Thing & Wh& kya && k-i-s se  \\
Thing & Indef & kuch && kuch se & kuch-kuch & kuch-kuch se\\\hline

\end{tabular}
\end{example}

\includegraphics[height=12cm]{pronouns.butt.king.2004.172.eps}

\includegraphics[height=9cm]{pronoun.forms.shukla.2001.195.eps}

\subsection{The oblique is predictable and obligatory}

\subsection{The oblique is a suffix not a postposition or a clitic}

In the next section I will provide arguments from Butt \& King (2004) that the oblique is a suffix, while the case postpositions are not.

\begin{example}(a) Case postpostions =ko can scope over conjoined NPs  (Butt \& King 2004:174)
\gll yasin=ne [\ipa{kUtt}-e or \ipa{g\textsuperscript{h}o\.*r}-e]=ko dek\textsuperscript{h}-a \ipa{hE}
Yassin.M.Sg=ERG dog-M.Sg.OBL and horse-M.Sg.OBL=ACC see-Pfv.M.Sg be.Prs.2.Sg
\glt `Yassin saw the dog and the horse.'
\glend
(b) The oblique cannot scope over conjoined NPs
\gll *[[\ipa{kUtt} or \ipa{g\textsuperscript{h}o\.*r}]-e]=ko
dog-M.Sg.OBL and horse-M.Sg.OBL=ACC
\glt `the dog and the horse'
\glend
(c) The oblique cannot scope over conjoined NPs
\gll *[[\ipa{kUtt}-a  or \ipa{g\textsuperscript{h}o\.*r}]-e]=ko
dog-M.Sg.OBL and horse-M.Sg.OBL=ACC
\glt `the dog and the horse'
\glend

\end{example}

\begin{example}(a) The focus particle3 can intervene between the NP and the case postposition  (Butt \& King 2004:174)
\gll \ipa{Us}=hi=ne kam ki-ya
3.Sg=Foc=ERG work.M.Sg.Nom do-Pfv.M.Sg
\glt `That one himself/only did (the) work.'
\glend
(b) The focus particle cannot intervene between the NP and the oblique
\gll \ipa{kUtt}-hi-e
dog-Foc-OBL
\glt `That one himself/only did (the) work.'
\glend
\end{example}


The -ko is a clitic, not a suffix. how about ne and other postpositions?


\section{Previous Discussion of the Oblique}

\subsection{Non-Discussion}
These properties have been described in Hindi/Urdu grammars (Kachru (1965), Kachru (1980:26), Pray (1970), McGregor (1972), Hook (1979), and in LFG literature, Mohanan (1994), Butt \& King (2004)).

Yet, so far the GB/P\&P/Minimalist literature on Hindi/Urdu case and agreement discusses only postpositions as case, overlooking the oblique forms. The oblique is not glossed in Mahajan 1990, Kidwai 2000, Davison 2004 and Dayal 2003, 2004.

\begin{example}jaane kaa (Mahajan 1990:160)
\gll Raam-ne Mohan-ko jaane~ka vaadaa diyaa
Ram-ERG Mohan to~go~gen promise gave
\glt `Raam promised Mohan to leave.'
\glend
\end{example}

\begin{example}larke-ko (Kidwai 2000:65)
\gll \ipa{sitA}-ne ek \ipa{la\:rke}-ko \ipa{p@s@snd} \ipa{kiyA}
Sita-ERG a boy-DAT liking did
\glt `Sita liked a boy.'
\glend
\end{example}

\begin{example}is laRkee-koo (Davison 2004:209)
\gll maiN-nee is-laRkee-koo deekh li-yaa
I-ERG this~boy-DAT see take-Pfv-M.Sg
\glt `I saw this boy.'
\glend
\end{example}

\begin{example}apne kamre meN (Dayal 2003 D23)
\gll puure din maiN-ne apne kamre meN kitaab paRhii
whole day I-ERG self's room in book-F read-F.SG
\glt `The whole day I read books in my room.'
\glend
\end{example}

\begin{example}kamre meN(Dayal 2004:403)
\gll kamre meN cuuhaa ghuum rahaa hai
room in mouse moving is
\glt `A mouse is moving around in the room.'
\glend
\end{example}

Bhatt 2005 includes it in his glossing on gerunds but not on nouns, and doesn't discuss its properties.

\begin{example}larke-ne Sita-se [kitaab parRh-ne]-ko (Bhatt 2005:780)
\gll ek-bhii \ipa{la\.*rke}-ne Sita-se kitaab \ipa{pa\.*rh-ne-ko} nah\~{i:} kah-aa
one-Psi boy-ERG Sita-INST book-F go-Inf-OBL-DAT Neg say-Pfv
\glt `Not even a single boy told Sita to read the book.'
\glend
\end{example}

Only Mohanan 1994 includes the gloss and a discussion of its appearance.

\begin{example}bacce-ke liye(Mohanan 1994:62)\\
NN = NonNominative
\gll bacce -ke liye
child(NN) GEN(NN) for
\glt `for the child'
\glend
\end{example}

\subsection{Previous accounts}

Previous accounts of the oblique have discussed the oblique in terms of an inflectional paradigm in the lexicon. In this paper I will show that a uniform analysis of the oblique's presence can be provided if we adopt a theory of abstract case features which are determined in the syntax, and phonological realization of the case features are determined in the morphology (Legate to appear). This understanding of the oblique is may prove useful in analyzing Hindi/Urdu case and agreement.

The oblique has been discussed in the LFG framework; both LFG accounts locate the oblique formation in the lexicon.

\begin{example}Mohanan 1994:\\
\begin{itemize}
\item ``just as stems carry information such as SINGULAR or PLURAL, they also bear the case features NOM, NONNOM, or VOC.” (Mohanan 1994:61)
\item ``NONNOM stems must take a case clitic after them” (Mohanan 1994:62)
\end{itemize}
\end{example}

\begin{example}Butt \& King 2004:\\
\begin{itemize}
\item The oblique is an agreement inflection on the complement of a K head (case postposition)
\item Butt \& King see ``this remnant of the Sanskrit system as ensuring synchronic morphological wellformedness: if the noun is in the oblique form, then modifying adjectives must also be in the oblique form.” (Butt \& King 2004:168)
\item The older morphological affixes (such as the oblique) may be part of the case system in other South Asian languages. (Butt \& King 2004:173)
\end{itemize}
\end{example}
However, researchers in LFG are calling for a development in their framework to deal with case.

Butt (2005) notes that  ``the differing realizational possibilities for case markers … one could argue about whether all of these different morphosyntactic creatures should all be lumped together under the label case … The overt realization of case must be dealt with by some component of the theory, however, that component is often left underspecified.” (Butt 2005:11)

So, a non-lexical analysis of the `direct' and `oblique' form may be brewing in the LFG literature.

\section{Proposal for case realization in Hindi/urdu: making the Nitty gritties explicit}

Proposal: the oblique is a case marker. The oblique is obligatory and uniform, its morphological realization is conditioned by the noun's class. This is no different from the accusative in English (which is found only on pronouns) and cases in German (which appear differently on masculine, feminine and neuter nouns).

\begin{example}The framework I adopt:\\
Case assignment: Pesetsky \& Torego 2001, Frampton \& Gutman 2002\\
Feature spreading: Vuchic 1993, Frampton \& Gutmann 2006, Kratzer 2006\\
Syntactic heirarchy in DPs: Exploded D, PhiP, Exploded P\\
Morphological realization: Marantz 1997, Legate to appear\\
\end{example}

\subsection{Case assignment}
I assume that the abstract case is assigned by a postposition to its DP complement (Pesetsky \& Torego 2001, Frampton \& Gutman 2002).

\begin{example}Case assignment\footnote{There is no evidence of a D\textsuperscript{o} in Urdu, I leave open the option of a DP or an NP for the examples in this
paper.}
\Tree {
 & PP\B{dl}\B{dr}\\
DP/NP\B{d} &\leftarrow[CASE]& se\Below{from}\\
\ipa{gad\textsuperscript{H}}e\Below{donkey-OBL} && \\
}
\end{example}


The morphology receives the bundles of features and realizes the bundles as faithfully as possible given the resources of the lexicon (Marantz 1997, Legate to appear).

\begin{example}Spell out\\
~~~~~~~~N\textsuperscript{o}\\
$\left[ \begin{array}{l}
DONKEY\\
+masc\\
+sing\\
+case\\
\end{array} \right] \Rightarrow gad^he$
\end{example}

\subsubsection{Multiple realization of same case}

Like case in many languages (German for example), the oblique appears multiply on all elements of the DP which have an oblique form.\footnote{I expect feature sharing to happen whenever there is a modifier, modifiee relationship (predicate modification, rather than functional application)}


\begin{example}Multiple Obliques in Hindi/Urdu\\
\gll us   bare   gadhe   se  le lo
that.OBL  big-OBL  donkey-OBL  from   take-Subj  take-2.Sg
\glt`Take it  from  that  big donkey.'
\glend
\end{example}

\begin{example}Multiple Accusatives in German\\
\gll Nimm   den Schemel   f\"{u}r  diesen   groSSßen grauen  alten   unbehaarten   ruhigen sitzenden  Elefanten   dieses Mannes
take-2.Sg.Imper the-OBL stool.OBL  for  that-OBL  big-OBL grey-OBL  old-OBL  un-?-hair-OBL  quiet-OBL  sit-?-OBL  elephant-OBL    that-GEN man-GEN
\glt`Take  the stool for this big grey old quiet hairless sitting elephant of the man.'
\glend
\end{example}

This structural case can be realized on pronouns, adjectives, nouns (in),  the genitive postposition, postpositions (in ), gerunds (in  and ),  these are the same grammatical categories as in German ().\footnote{If we think that the -en on participles is the same thing.}

\begin{example}
\gll[dabbe~ke    nice    tak]
box-OBL=GEN-OBL  under-OBL  until
\glt `up  on the underneath of  the box.'
\glend
\end{example}

\begin{example}
\gll [raam~ke   baiThne~par]    mãã~ne  us~ko    khaanaa  diyaa
raam=GEN-OBL  sit-NONFIN-OBL=ON  mother=ERG  3sg.OBL=DAT  food   give-Pfv
\glt `When Ram sa  down, mo her gave him food' (Mohanan 1994:78)
\glend
\end{example}

\begin{example} ~[Lataa-ji-ke ye gaanaa gaa-ye ho-ne ] se (Bhatt 2005:765)
\gll Ashaa=ji=ka ye gaanaa gaayaa honaa [Lataa=ji=ke ye gaanaa gaaye hone]=se zyaadaa mumkin hai
Ashaa=Hon=GEN-M.SG this.M.Sg sing-Nmlz-M.SG sing-Pfv-M.Sg be-NonFin-M.Sg [Lataa=Hon=GEN-OBL this.M.Sg.OBL sing-Nmlz-M.Sg sing-Pfv-M.SG.OBL be-NonFin-M.Sg.OBL]=INSTR more possible be-Prs.3.Sg
\glt `Ashaa-ji's having sung this song is more likely than Lataa-ji's having sung this song.'
\glend
\end{example}



It is clear that the case and gender features are shared within the projection. One way to accomplish this feature sharing is through feature spreading (Vuchic 1993, Frampton \& Gutman 2006, Kratzer 2006). This must be accomplished in a two-step process.


\subsubsection{Upward feature Percolation/Projection}

It is commonly assumed that the projecting sister projects its features to the mother node. As the possessor (GenP) is optional I assume it is adjoined to the NP, the same is true of the AdjectiveP. Cinque 2005 considers the DemontrativeP is also adjoined. As all elements are adjoined the features of donkey are projected.

\begin{example}The features of the noun are projected/percolated through merger (Frampton \& Gutman 2006)

\Tree{
 & NP\B{dl}\B{dr}\\
DemP\Below{DISTAL} && NP\B{dl}\B{dr}\\
&AdjP\Below{BIG} && NP\\
 }
$\left[ \begin{array}{l}
 +masc\\
 +sing\\
 \end{array} \right]$

\end{example}

\subsubsection{Downward feature sharing}
this migth be controversail, could be the divide between synatx and morphology, syntax is bottom up, morphology is top down...

\begin{example}The case feature of the complement must be spread down to all elements in the DP/NP (might be controversial, need to read more literature)
\end{example}



Finally, the feature bundles are spelled out as words using the lexical item which matches the most of the features. In this case, the demonstrative us `that' isn't specified for masculine or feminine, as there is no competing demonstrative, it is inserted. The adjective bare `big' and the noun gadhe `donkey' are not specified for singular or plural, as there are no competing forms for singulars these are inserted.

\begin{example}Lexical Insertion
\end{example}

\subsection{Proving the oblique is obligatory/there when you don't see it}

So far I have shown only masculine examples where the oblique is seen overtly. In the next section I will show that the oblique is regular and required, its appearance on some words and not on others is due to its morphological availability for certain words and not others.

In order to prove that the oblique is obligatory in the examples below, I will use a demonstrative, which show overtly the 4 way distinction in number and direct/oblique marking, and an adjective which show the 2 way distinction in gender marking.

\begin{example}Hindi/Urdu demonstratives\\
\begin{tabular}{|l|l|l|l|l|}\hline
& \multicolumn{2}{c|}{Singular} & \multicolumn{2}{c|}{Plural}  \\\hline
& Direct & Oblique & Direct & Oblique \\\hline\hline
Proximate & ye & is & ye & in\\
Distal & vo & us & vo & un\\\hline
\end{tabular}
\end{example}

\begin{example}Hindi/Urdu Adjectives\\
\begin{tabular}{|l|l|l|l|l|l|}\hline
& \multicolumn{2}{c|}{Singular} & \multicolumn{2}{c|}{Plural}  &\\\hline
& Direct & Oblique & Direct & Oblique &\\\hline\hline
Masculine & bara & bare & bare & bare & `big'\\
Feminine & bari & bari & bari & bari & \\\hline
\end{tabular}
\end{example}

The morphology spells out each of these feature bundles given the resources available. If the morphology is given a masculine noun which has a lexical entry which realizes the oblique feature, the noun will surface in the oblique form, as the adjective bare `big' does in (). If the masculine noun does not have an oblique form, the noun will surface as the default form, as the noun mez `table' does in ().

\begin{example}Masculine oblique surfaces as the direct form if there is no oblique
\gll(a) [wo   bara   mez]   gir  reha   hɛ
~ that.DIR big-MSG.DIR  table.MSG.DIR  fall IMPERF-M  PRES-3SG
\glt `That big table is falling.'
\glend
\gll (b) [us   bare   mez   se]  le   lo
~ that.OBL  big-OBL.DIR  table.MSG.OBL  from  take-SUBJ take-2nd.sg
\glt `Take it from that big table.'
\glend
\end{example}

If the morphology is given a feminine noun with an oblique feature, it will surface the default feminine form since feminine nouns do not have a lexical entry for the oblique.

\begin{example}Feminine oblique surfaces as the direct form if there is no oblique
\gll (a) [wo   bari   kitaab]  gir rehi  \ipa{hE}
~ that.DIR  big-FSG.DIR  book.FSG.DIR  fall IMPERF-F  PRES-3SG
\glt`That big book is falling.
\glend
\gll (b) [us   bari   kitaab   se]  le   lo
~ that.OBL  big-FSG.OBL  book.FSG.OBL  from  take-SUBJ  take-2nd.sg
\glt `Take it from that big book.'
\glend
\end{example}
31.Spell out for feminine nouns


Thus the appearance and non-appearance of the oblique case does not stop it from being a regular case if one accepts Legate's (to appear) proposal that abstract case is different from overt morphological case. This shouldn't come as a surprise, we assume that ``John” is marked accusative in the sentence ``I see john,” even though there is no way of realizing accusative on John. We know that the accusative is there because when you replace John with a pronoun, the pronoun must be accusative ``him” not the nominative ``he.”

\begin{example}English abstract case is also realized differently on different words:\\
Compare John/John and he/him.\\
\begin{tabular}{lllll}
a) & John & sees & Bill. & \\
~ & NOM & ~ & ACC\\
b) & I & see & John.\\
~ & NOM & ~ & ACC\\
c) & I & see & him.\\
~ & NOM & ~ & ACC\\
d) & *I & see & he.\\
~ & NOM & ~ & NOM\\
\end{tabular}
\end{example}

\section{Summary}

In this paper I have shown that it is possible to formalize the distribution of the oblique noun forms using a syntactic analysis. The formalization of the oblique in the syntax should be useful in analyzing the problematic Hindi/Urdu case and agreement.


\section{Ramifications of this paper}

\subsection{Hindi/Urdu postpositons aren't realizations of abstract case}

If postpositions in Hindi/Urdu are assigning/checking a case, then they must be functional heads in the syntax.

This is only compatible with an analysis of the case postpositions as something equivalent to a PP or a KP7. This can be argued for independently on the grounds that Hindi/Urdu postpositions have regular semantics, and introduce only one type of theta role (Butt \& King 2004).

This provides a further argument that the case postpositions are certainly not realizations of ``structural” case. Structural case is thought of a syntactic case feature, but a syntactic feature can't assign another case.

If case postpositions in Hindi/Urdu are functional heads in the syntax, this explains why passivization and argument raising retains the postposition, if we consider that passives are transformationally derived from actives. The highest maximal projection is the PP, so that must be raised, postposition and all.

Look into the ``to him” for English passives and compare the two passive and active sentences.

\subsubsection{Except the ergative...}
The ergative postposition is different from the others, as the pronouns are in the direct not the oblique form. I make no claims about the ergative.


\section{Issues Hindi-Urdu case \& agreement which are affected by this paper}

In this section I will briefly outline areas where the case and agreement of Hindi/Urdu is claimed to behave differently from the crosslinguistic typology.

\subsection{Contra Bhatt 2005, Hindi-Urdu case and agreement can't be two sides of a coin}

Case and agreement are ideally two sides of a coin, the head probes for interpretable φ-features to check its uninterpretable φ-feature, in turn, the YP has uninterpretable case features are checked by the interpretable θ-role (Chomksy 1995).  Alternatively, the functional head assigns abstract case features to its complement, and the complement shares its φ-features with the functional head. (`Crash Proof Syntax' Frampton \& Gutmann 2006) In either a checking or an assignment approach the features are realized as person/gender agreement on the function head, and as a case marker on the complement.

\begin{example}Case and agreement are two sides of one relationship
\end{example}

Mahajan 1990: Hindi/Urdu verbs never agree with a case marked DPs. If we assume that the verb is assigning case, then the verb should agree with that DP. However, for Mahajan, the postpositions are case markers. If those are postpositions rather than case, then certainly the verb doesn't agree. My proposal derives that they don't agree.

\subsection{Contra Bhatt 2005: Hindi/Urdu is an exception to Burzio's generalization}

\begin{example}Accusative Ko is retained in passives (Bhatt 2005:782)\\
but not all passives?
\end{example}

This is not an exception if -ko is not a structural case.

Bhatt 2005: Hindi/Urdu infinitives license structural case.
37.The complement of an infinitive must be accusative. (Bhatt 2005:782)

This is not an exception if -ko is not a structural case.

\subsection{Contra Legate 2007: Hindi/Urdu has ``aggressive agreement”}

Little v will agree the subject, but if it finds no eligible DP, it will search down and agree with the object


``I propose that the pattern of agreement found in Hindi is similar in essentials to Niuean in that DPs that bear inherent ergative Case do not trigger agreement. Where the languages differ is this:


``in Hindi, after the inherent Case-marked DP fails to trigger agreement, T continues to search down the tree for a DP that may trigger agreement, i.e. a DP with structural Case. In (37c), T finds the accusative object, which then triggers subject agreement, even though it has no other relationship with T. I refer to this as aggressive agreement.31 `` (Legate p17) ``Evidence for aggressive agreement comes from two sources. First, we find that (pseudo)-incorporated nominals trigger agreement in Hindi, even though such DPs crosslinguistically lack Case. (See for example, Baker (1988) on the lack of Case on incorporated nouns and Massam (2001) on the lack of Case on pseudo-incorporated NPs.) Thus, these DPs trigger agreement without bearing nominative Case or raising to [spec, T], simply based on closest c-command.” (Legate p17)
This is not the only place where Hindi shows object agreement, object agreement  is only in gender, never in person, where as subject agreement is both gender and/or person. This requires a look into the morphological realization and syntactic relationships behind agreement in Hindi/Urdu.


\section{For the future: Distribution of person agreement vs gender agreement in Hindi/Urdu appears to be systematic}

The the distribution of agreement in Hindi/Urdu appears to be systematic, object agreement is in gender, never person. Subject agreement may be either gender or person depending on the verb. Lexical verbs show gender, light verbs show person. This agreement pattern might indicate a systematic difference either in the \textit{agreement relations} between lexical verbs and light verbs, or more simply, a difference in \textit{morphological availability} of person morphology for lexical verbs and light verb stems.

\appendix

\section{References}

\begin{reflist}

Gerund and Gerundive in Latin
Miller, DGary
Diachronica, Vol. 17, No. 2, 2000, pp. 293-349.
-latin gerunds also agreed with the object read more. this doesnt make sense if a gerund is a noun (with default gender reatures) but does make sense if its more like an adjective/verb/predicate.


Bhatt, R. 2005. Long distance agreement in Hindi-Urdu. Natural Language and Linguistic Theory 23.4:757-807.

Butt, M. \& T. H. King. 2004. The Status of Case. In Clause Structure in South Asian Languages, ed. V. Dayal \& A. Mahajan. Dordrecht: Kluwer.

Chomsky, N. 1970. ``Remarks on Nominalization,” in Studies on Semantics in Generative Grammar, The Hague: Mouton, 1972, 11-61.

Chomsky, N. 1995. The Minimalist Program.

Chomsky, N.. 2000. Minimalist Inquiries: The Framework. In Step by Step. Essays on Minimalist Syntax in honor of Howard Lasnik, ed. Roger Martin, David Michaels and Juan Uriagereka, 89-155. Cambridge, MA: MIT Press.

Chomsky, N. 2001. Derivation by Phase. In Ken Hale: A life in language, ed. Michael Kenstowicz, 1-52. Cambridge, MA: MIT Press.

Corbett, Greville G.
Agreement /
Cambridge, UK ; New York : Cambridge University Press, 2006.
xviii, 328 p. : ill. ; 26 cm.
Morris Library  P299.A35 C67x 2006 Normal Loan

Frampton, J. and S. Gutmann. 2002. Crash-Proof Syntax. In Derivation and Explanation in the Minimalist Program, ed. by Samuel D. Epstein and T. Daniel Seely, 90-105. Oxford: Blackwell Publishing.

Frampton, J. \& S. Gutmann. 2006. 'How Sentences Grow in the Mind: Agreement and Selection in an Efficient Minimalist Syntax'. In C. Boeckx (ed.), Agreement Systems. Amsterdam: John Benjamins, 121-157.

Halle, M. and A. Marantz 1993. ``Distributed Morphology and the Pieces of Inflection,” in K. Hale and S.J. Keyser, eds., The View From Building 20, Cambridge, Mass.: MIT Press, 111- 176.

Legate, J. To appear. Morphological and Abstract Case.

Mahajan, A. 1990. The A/A-bar Distinction and Movement Theory. MIT Dissertation.

Masica, C. 1991. The Indo-Aryan Languages. Cambridge: Cambridge University Press. PK115 .M37 1991 

Mohanan, T. 1994. Argument Structure in Hindi. Stanford, CA: CSLI Publications.

Payne, J. 1995. ``Inflecting Postpositions in Indic and Kashmiri.” in Double Case, Agreement by Suffixaufnahme. Plank, F. (ed). Oxford University Press.

Pesetsky, David \& Esther Torrego. 2001. T-to-C movement: Causes and consequences. In Ken Hale: A Life in Language, ed. Michael Kenstowicz, 355-426. Cambridge, Mass: MIT Press.

Shukla, S. 2001. Hindi Morphology. München: Lincom Europa.

Vuchic, R. 1993. A study of Noun Phrase agreement in French as a second language: an autosegmental model. University of Delaware Dissertation.

http://www-uilots.let.uu.nl/ltrc/agreement.htm
Agreement Database
http://www.ling.lancs.ac.uk/staff/anna/anna.htm

\end{reflist}



\section{A family tree of Hypothesis B}

\section{Arguments for Case in the Syntax over Case in the Lexicon}

``Chomsky (1965:221-222, footnote 35) does note that he assumes case marking to be assigned at the level of phonological realization.” (Butt 2005:29)

Both Mohanan 1994 and Butt \& King 2004 take an approach where morphological processes take place in the lexicon (the ``lexicalist” position).

Falk argues against having case in the lexicon in his proposal for an additional architecture projection in LFG for case: ``the premise is that the syntax must be able to specify the Cases that surface in the language. In any language in which there is an accusative Case, the syntax must be able to specify it. The same is true of ergative Case. This seems to us to be an uncontroversial assumption.” (Falk 1997)


\subsection{Arguments against the (strong) lexicalist position}
Reading list \\
Chomsky 1970 \\
Marantz 1997\\

\subsection{Arguments for inflectional morphemes in/out the syntax}
Reading list\\
?

\subsection{Arguments for derivational morphemes in/out the syntax}

Reading list\\
?

\subsection{Conclusion of lexical vs syntax}
As an alternative to ``lexicalism,” I adopt Late Insertion of functional items (Halle \& Marantz 1993, Marantz 1995, Chomsky 2001) where phonological material is inserted in the morphology to realize bundles of syntactic features. Under this approach the appearance of the oblique must be explained. This also allows a happy medium between non-uniform paradigm like morpheme class dependence and obligatory appearance (productivity) of `oblique.'


\section{Case in the syntax: feature vs. head}

\subsection{What is a terminal node?}
\subsubsection{Seperatable morphemes?}
\subsubsection{Morphosyntactic features?}
\subsubsection{Phonological words?}
\subsubsection{Morphosyntactic feature tree as a phase, resulting in a phonological word as a terminal}
\subsection{Other potential arguments, but no one to cite yet:}

\subsection{Dependant on morphological class, not grammatical category}

The realization of the oblique is dependant on morphological classes and gender. This is indicative of inflection rather than a functional head (although some functional heads show idiosyncratic realization (need to check Tim's talk and Hayes).

If the oblique were a functional head then it would select for a syntactic category rather than morphological class.

\subsection{Summary of feature vs head, an open question}

Summary: the oblique is a suffix, which is realized differently for different noun classes.

It's a suffix, but is it a feature or a functional head?
A structural case that is assigned by P\textsuperscript{o}



\section{Is the oblique actually the accusative case?}

\subsection{Is this structural case only on objects of prepositions?}

Legate (to appear) argues that abstract structural accusative case in Hindi which is assigned by vº is the morphological default, null. This is needed for her analysis reducing absolutive to nominative or default in different languages. This is true in the verbal domain, but given the above data there does appear to be an overt realization of abstract structural case, that which is assigned by postpositions to their complements. As this these are postpositions rather than vº  this doesn't challenge Legate's claim, rather Legate is most likely correct in analyzing Hindi/Urdu case and agreement as non-standard in that it lacks a morpheme specific to structural accusative case.

\subsection{No, there are other obliques that don't have a postposition}

Now that we consider that the oblique is a case, we need to find case assigners for its appearance.

Yet, there are obliques which have no obvious postposition to assign case

\subsubsection{The oblique as a locative}
39.The oblique as a locative (Butt \& King 2004:168) 8


\subsubsection{The oblique in inalienable possession}
40.The oblique in inalienable possession  (Mohanan 1994:178)


41.The oblique (maybe) in an adjunct under the  emphatic particle, not m.sg because the gen on the subject is also marked as oblique. What dialect is this?


Davison 2004:202

\subsubsection{The oblique under the light verb 'allow'}
42.The oblique under allow (Bhatt 2005 :778)
[Sarosh-ko gaaRii chalaa-ne] di-i

[peR kaT-ne] di-ye (Bhatt 2005:795)

\subsubsection{The oblique under a conditional}
43.In a conditional (collected from a web forum)
  Agr mere   pice  kutta  bhonkne   laga
  if 1st-GEN-OBL behind-OBL dog  bark-NONFIN-OBL start-M.SG
to  main  zurud   bhag  jaoongi
then  1sg  certainly  run  go-1sg=FUT
  `If a dog started barking after me then I would certainly run.'




Like multiple obliques with in a DP, the oblique can appear on conjoined verbs

44.Multiple marking on conjuncts (collected in a web forum)
main aap ko
1.sg 2sg.FORMAL=DAT
danda dhondne   ya phir  jota oTarne
stick look.for-NONFIN-OBL or even  shoe take.off-NONFIN-OBL
duun  gaa  kya?
Allow-1.SG FUT Q

It is a structural case, but is it a structural case assigned what. What unifies postpositions, locatives, inalienable possessives, permissives and focus?
An existential operator…?


\subsubsection{The oblique on subjects of certain transitive verbs}

``The subjects of eligible transitive verbs were marked as oblique” (Butt \& King 2004) (19)


\end{document}
\section{What is agreement, can Hindi/Urdu agreement become normal if we consider (following Kratzer 2006) that gender agreement is different?}

Butt 1995: there is a great deal semantic and syntactic evidence that nouns like girl in  ``girl-looking” are incorporated, yet the noun and verb agree.


38.``pseudo-incorporated' nouns show agreement




This agreement has been considered a sign that the noun cannot be pseudo incorporated by Bhatt 2005. Nouns which are incorporated are not case marked and don't show agreement with the verb (Baker 1988, Massam 2001).


However, the pseudo-incorporated noun and verb agree only in gender. Kratzer 2006 argues that sharing gender features is done under nonrestrictive modification, which is precisely the semantics for pseudo-incorporated nominals. I would like to argue that agreeing in gender is a sign of a modifier rather than an argument. And so this would confirm that they are modifiers, not arguments as has been suspected, the only barrier to this analysis before was that there was still agreement.




\section{For Ben: Two kinds of abstract case features, structural and inherent, not really relevant}

Inherent case is an idiosyncratic case which depends on the lexical item which assigns the case.

If different postpositions assigned different case on a par with the oblique then it would be worthwhile investigating whether the oblique is inherent. German postpositions assign different cases, the postposition für `for' assigns/checks accusative, while von `from' assigns/checks dative.

However, in Urdu all postpositions assign/check oblique, as all members of a syntactic category assign/check the same case, it seems non-sensical to entertain that that case is inherent; ie an idiosyncratic case which is characteristic of a particular lexical item.

Tests for inherent case come from the verbal domain rather than the PP domain.

One test is whether the case-marked NP/DP retains its case when passivized. What the corresponding action a postposition's compliment would be escapes me. (Ben argues against A-movement as a test since non-A-moved objects loose their structural case in Icelandic passives. Ben says that if the markedness theory of ergative marking were correct, an inherent case would be a case where a co-argument is demoted, but no grammatical function change takes place. In the antipassive the object is demoted to an oblique while the subject remains a subject. Postpositions don't appear to have co-arguments in Urdu, so this test for inherent case is not possible.)

Since I think that the inherent/structural case is irrelevant this shouldn't be a problem.

My claim: postpositions are functional heads which participate in event structure, they introduce arguments. In this way they are ``semantic case” if anything. They are not ``structural” cases, at least not in terms of the assignment/checking of ``features”. They must be present as functional heads, in the syntax, because they assign a case (oblique). The oblique itself is a ``structural” case in that it is a feature in the syntax, not a head.

\section{No -e}
Kidwai 2000:165


\section{Masculine nouns}
Sona, sone gold Butt \& King 2004:164

Shukla 2001:6


\section{3rd singular -e}
Butt \& King 2004:165

Kidwai 2000:43


\section{Participles (with varying gender)}
Participles (this is not oblique as it it should become -i with a fem subject) (Mohanan 1994:182)

\section{Can this be in the oblique, us aurat instead of voh aurat}




\section{Plural -e}

45.(Bhatt 2005:787)

46.Is that certainly plural in this example? (Bhatt 2005:796)


\section{Forms of se - sa,si,se?}

(Butt \& King 2004:170)

47.Kon saa (Mahajan 1990:40)

48.Kon sii (Mahajan 1990:

49.Kon sii (Mahajan 1990 :117)


\section{Adverbs (might show acc like in quechua)}

50.Dhire dhiire `slowly' (Mahajan 1990:123)

51.Kese (mahajan 1990)

52.phir se (bhatt 2005:766)

53.Jaldii jaldii/jaldii se `quickly'

54. baRi caturtaa se `cleverly'

55.kitne `how many'



\section{Vaalaa,vaalii,vaale}

56.In a possessive (Mahajan 1990:37)


