\documentclass[12pt]{article}
\usepackage{covington}
%\usepackage{fullpage}
%\usepackage{tree-dvips}
%\usepackage{qtree}
%\usepackage{tipa}
%\usepackage{ulem}

%\let\ipa\textipa %to use \ipa rather than \textipa

\author{Julie C. Sedivy}
\title {Pragmatic versus Form-based Accounts
of Referential Contrast: Evidence for Effects
of Informativity Expectations}
\date{March 08 2006}
\begin{document}
\pagenumbering{arabic}
\maketitle

\begin{itemize}
\item [model-theoretic] approaches to
linguistic meaning, which attempt to formally characterize the relationship
between structural aspects of language and a representation of its meaning in
terms of information conveyed about properties and relations of entities
instantiated in an abstract model of the world (for an extensive introduction,
see Dowty et al., 1981).

\item [discourse context] typically focused on information that is linguistically instantiated in a
discourse context preceding some target sentence and have measured the
impact of this contextual information on the reading time of sentences that
are syntactically ambiguous at some point in the sentence.

\item [highly incremental] interpretation
occurs in a highly incremental fashion in time and that on-line language
processing is highly sensitive to specific properties of the model
against which the linguistic input is being interpreted.


\item [mapping before offset] mapping of referential expressions to a visual
model is initiated on the basis of extremely partial information, beginning
well before the offset of the referring word,

\item [information effects] are seen from lexical frequency (Dahan et al.,
2001), prosodic information (Dahan et al., 2000; Sedivy et al., 1995), verbbased
semantic constraints

\item [Presupposition ] a varied
set of linguistic expressions which signal that certain information is assumed
to be taken for granted and backgrounded in a discourse model. presuppositions can be lexical (know vs believe) or constructional (clefts)Presuppositions are typically not cancellable; though they can be accommodated into a
context in which the presupposition has not previously been entered into the
discourse, the discourse sounds anomalous if accommodation of the presuppositional
material is inconsistent with the previous discourse.
\end{itemize}

\section{Temporaly ambiguous sentences with modified noun phrases}

\section{Postnominal Modifiers}
Some researchers claim  modifier phrases are used to distinguish when there are multiple entities denoted by the head noun.

\section{Focus Operators}

The semantic effect of
focus in the sentence above is to ensure that beyond simply asserting that
John smokes cigars, the sentence also establishes a distinction between John
and an implicit, contrasting set of individuals and asserts that none of these
individuals smokes cigars.

The presence of a focus operator
dramatically reduced the difficulty normally associated with temporarily
ambiguous reduced relative clauses.

difficulty reemerged when subjects were presented with a prenominal modifier such as 'wealthy' in Only wealthy businessmen loaned money at
low interest were told to record their expenses, Suggesting that when the prenominal modifier offered an opportunity for setting up a contrast set

\begin{example}
 a. All of the secretaries and accountants were made to take a
tough computing course.\\
b. All of the secretaries in the company were made to take a tough
computing course.
\end{example}

\begin{example}Target sentence:\\
\item Only the secretaries prepared for the exam and earned significant
pay raises.
\item Only the secretaries prepared for the exam passed and earned pay
raises.
\end{example}

similar manipulations of contrast involving sentences that were not
marked for focus did not show any contextually based effects.

\section{Intonational Focus}

contrastive focus can appear in a sentence in the absence of an explicit focus operator. Typically, contrastive focus is marked intonationally
by a L+H* accent

while a contrast set may be evoked, the relationship between the focused and
contrasting entities is not explicit and may be determined by context. For
instance, uttering the sentence JOHN smokes cigars may convey a correction
of someones mistaken assertion that someone else smokes cigars, call attention
to Johns smoking habits relative to other entities, or express a contrast
in knowledge pertaining to other individuals versus John (e.g., as in I dont
know whether anyone else likes cigars, but JOHN smokes cigars)


\section{Adjectival Modifiers}

\section{A linguistically mediated versus pragmatically inferential account of referential contrast}

\section{Conclusions and Further Questions}

\end{document}