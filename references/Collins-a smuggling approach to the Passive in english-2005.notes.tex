\documentclass{article}
\usepackage{covington}
\usepackage{fullpage}
\usepackage{tree-dvips}
\usepackage{qtree}
\usepackage{tipa}
\usepackage{2up}
\let\ipa\textipa %to use \ipa rather than \textipa

\author{Summarizer: Gina Cook}
\title {Summary of:\\``A Smuggling Approach to Passive in English"\\ by Cris Collins\\ Syntax 8:2, August 2005, pgs 81-120}
\date{for Topics in Syntax, October 26th  2005}
\begin{document}
\pagenumbering{arabic}
\maketitle

% \section*{Abstract}
% Collins combines:
% \begin{enumerate}
% \item Principals and Parameters (P\&P) analysis\\
% -no specific rules, no downward movement
% \item Chomsky's 1957 \emph{Syntactic Structures}.\\
% -the arguments in the passive are generated in the same
% positions as they are in the active
% \end{enumerate}

\section{Introduction}
\begin{description}
\item [Severe problem with P\&P:] The external argument is generated in different positions for active and passive. 

\item [Solution from \emph{Syntactic Structures}: ]
external argument occupies the same underlying position in both active and passive
\end{description}


\section{Principles and Parameters Meets Syntactic Structures}
\begin{example}
P\&P structure:
\Tree [.IP [.DP \node{DP2}the book ] [.I' I [.VP [.VP [.V V EN ] \node{DP1}DP ] [.PP by John ] ] ] ]
\anodecurve[bl]{DP1}[bl]{DP2}{.6in}
%\abarnodeconnect{DP1}{DP2}{.8in}
\end{example}

\begin{example}
P\&P assumes the passive suffix -en absorbs
\begin{itemize}
\item [a.] accusative Case
\item [b.] the external $\theta$-role
\end{itemize}
This assumes that the passive suffix -en is itself an argument that is assigned Case and receives the external $\theta$-role. Colin rejects both these implied claims.

\end{example}

\subsection{The $\theta$-role of the by phrase is different with different verbs}
The $\theta$-role of the postverbal DP is
not agent but rather varies with the verb
 which suggests that \emph{by} does not assign a $\theta$-role.

\begin{example} 
Examples of by phrases with a variety of $\theta$-roles:
\begin{description}
\item [Agent] ``The book was written by John."\\
The doer of an action (under some definition must  be capable of volition).
\item [Experiencer]  ``Mary was respected by John."?? \\
The argument that perceives or experiences an event or state.
\item [Theme] ``That professor is feared by all students."??\\
The element that is perceived, experienced or undergoing the change of state
\item [Goal] ??\\
The end point of a movement.
\item [Recipient] A copy of Guns, Germs, and Steel has now been received by each member of the incoming class.??\\
A special kind of goal, found with verbs of possession (e.g., give).
\item [Source] A black smoke was emitted by the radiator.\\
The starting point of a movement.
\item [Location] ??\\
The place an action or state occurs.
\item [Instrument] ??\\
A tool with which an action is performed.
\item [Benefactive] ??\\
The entity for whose benefit the action is performed.
\item [Proposition] ??\\
The thematic relation assigned to clauses.
\item [idiom]\footnote{Earlier class discussion asks whether these NPs are really idiomatic.} ``Ted was bitten by the lovebug.''\\
Passive idioms are not supposed to be capable of assigning agent (according to Marantz.) 
\end{description}
\end{example}

\subsection{How does the DP after the verb get its $\theta$-role?}
If the theta roles that are supposedly assigned by \emph{by} are varied, how does the postverbal DP get assigned a $\theta$-role in the passive?


\begin{description} 
\item [Jaeggli (1986:590):] First, the passive suffix -en
absorbs the external $\theta$-role of the verb. Second, the passive suffix assigns the
PP headed by the preposition by the external $\theta$-role ($\theta$-role transmission).
Third, the $\theta$-role assigned to the PP percolates to the preposition by, and lastly
by assigns the external $\theta$-role to its DP complement.

\item [Baker, Johnson, and Roberts (1989:223):]  the passive suffix -en transmits it theta role through a nonmovement chain with the \emph{by}phrase similar to clitic doubling.
\end{description}

{\bf Cautionary Note} Are these claims equivalent enough so that if Collins shows a better alternative to Jaeggli then it follows the alternative is superior to Baker et al?


%In the remainder of the paper, Collins focuses on Jaeggli's analysis, stating that both analises suffer from similar problems.


\subsection{Problem with Jaeggli's analysis}

\begin{description} 
\item [Violates UTAH] the external argument in
the passive is assigned a $\theta$-role (via $\theta$-role absorption and transmission) in a
way that is totally different from how the external argument is assigned a
$\theta$-role in the active (in Spec,IP in the principles and parameters framework).
\end{description}

\begin{example}
\emph{UTAH} Uniformity of Theta-Assignment Hypothesis\\

Identical thematic relationships
between items are represented by identical structural relationships. In the Minimalist Program, the effects of UTAH follow from the fact that all $\theta$-role assignment is configurational, in the sense that each syntactic position (e.g., Spec,vP, complement V) is associated with a particular $\theta$-role (or set of $\theta$-roles).
\end{example}


\subsection{Collin's Proposal}
\begin {enumerate}
\item The external argument is merged into Spec,vP in the passive
 and in the active 
\item  Passive participle suffix and the past participle suffix are the same;  no morphological difference  in English.
\item \emph{By} makes
no semantic contribution, its presence must be forced by syntactic reasons.
\item The participle morpheme -en heads a PartP and that V
raises and adjoins to Part.
\item Part takes a VP complement and 
\item PartP is the complement of v 
\item Languages either use a participle (non-finite form) (English) or not (Kiswahili)
\item If a language has a passive morpheme it will be generated in VoiceP ($v$) 
\item If a language does not have a passive morpheme and uses a preposition like element, that element will be generated in VoiceP.
\item $v$ assigns the external $theta$-role
\item voice checks accusative case
\item Freezing does not hold for all types of movement\footnote{What is freezing?: ``It is important to note that smuggling derivations assume that Freezing
(Mu� ller 1998:124) does not hold for all types of movement, where Freezing is in Collins example
(34) Z [YP XP ] W <[YP XP ]>} "
\item Voice, not v, could be the strong phase head in the case of the passive 
\item Even when the external argument is not phonologically present it must be syntactically present for binding and licencing reasons
\item Comp, not Infl, checks null Case.
\item The empty category in the passives without a by-phrase is simply arbitrary PRO.
\item If VoiceP undergoes movement, everything
except Spec,vP must be evacuated.
\end{enumerate}

Structure: \Tree [.VoiceP \node{PartP2}PartP [.Voice' Voice [.vP DP [.v' \node{head1}v  [.PartP Part VP ]  ] ] ] ] 

%\Tree [.vP byP [  v [.PartP Part\\{[en]} [.VP V DP ]  ] ] ]

\begin{example} \label{derivation}
Derivation for ``The book was written by John."
\begin{enumerate}
\item John 

\item Merge external argument $\rightarrow$ [vP [DP John] [.v'  ...] ] 
\Tree [.vP [.DP John ] [.v'\\{...} ]  ] 

\item Merge Voice $\rightarrow$ [VoiceP by [vP [DP John] [v' ...]]]
\Tree [.VoiceP by [.vP [.DP John ] [.v'\\{...} ] ] ]


\item Merge be $\rightarrow$ [VP be [VoiceP by [vP [DP John] [v' ...]]]
\Tree [.VP be [.VoiceP by [.vP [.DP John ] [.v'\\{...} ] ] ] ]

\item Merge Infl $\rightarrow$ [IP Infl [VP be [VoiceP by [vP [DP John] [v' ...]]]]
\Tree [ Infl [.VP be [.VoiceP by [.vP [.DP John ] [.v'\\{...} ] ] ] ] ]

\item Internal Merge of [DP the book] into Spec,IP 
\Tree [.IP [.DP the book ] [.I' Infl [.VP be [.VoiceP by [.vP [.DP John ] [.v' v [.PartP Part\\{[en]} [.VP V ]  ] ] ] ] ] ] ]
\end{enumerate}

\end{example}

\section{PartP Movement in the Passive}


The derivation in (\ref{derivation}) yields the wrong
word order of a passive sentence
\begin{example}
 *The book was by John written. \\(Collins 9a)
\end{example}

\subsection{Four ways to move to get the right order:}

\begin{description}
\item [{[Spec v'] instead of [v' Spec]}] rightward specifier analysis makes the wrong
predictions about standard Barss and Lasnik (1986) c-command tests. It also wrongly predicts that a negative
quantifier in the by-phrase should license a preceding negative polarity item and that each in the by-phrase should license a preceding the other.
\item [Move PP rightward] rightward movement (extraposition) analysis of by-phrases makes the same wrong predictions about c-command.
\item [Move part head leftward] particles would have to incorporate to move with the part head in order to get the correct order. Prepositions stranded in pseudo-passives also need to be next to the participal. Collins assumes incorporation is unlikely.
\item [Move partP leftward] Gives the correct order with particles\footnote{particle are analyzed as being generated in the complement position of VP}, Collins 15b. ``*The argument was summed by the coach up." What is the motivation of moving the PartP?
\end{description}


Head to head movement
\Tree [.VoiceP [.Voice \node{head2}v Voice ] [.vP PP [.v' \node{head1}v [.PartP \node{Part}Part [.VP \node{V}V DP ] ] ] ] ]
\anodecurve[bl]{head1}[bl]{head2}{.4in}
\anodecurve[bl]{Part}[bl]{head1}{.4in}
\anodecurve[bl]{V}[bl]{Part}{.4in}
 %\anodecurve[b1]{head1}[b1]{head2}{.5in}%
 %\anodecurve[bl]{Part}[b1]{v1}{.4in}
%\anodecurve[bl]{V}[bl]{Part}{.4in}

XP movement
\Tree [.VoiceP \node{PartP2}PartP [.Voice' Voice [.vP PP [.v' \node{head1}v  \node{PartP1}PartP ] ] ] ] 
\anodecurve[bl]{PartP1}[bl]{PartP2}{.4in}

\subsection{What causes PartP to move?}
Collins draws an analogy between the licensing of a participle and structural Case. Both have the following features.

\begin{example}
 A participle (PartP) must be licensed\\
a. being c-selected by the auxiliary\\
b. moving to Spec,VoiceP.\\
\end{example}

\begin{example}
a. The auxiliary verb \emph{have} obligatorily c-selects for a participle.\\
b. Voice requires a participle (PartP) to move to Spec,VoiceP.
\end{example}

% \subsection{Selection of have and be??}
% 
% \begin{itemize}
% \item be selects for a variety of complements, voiceP included
% \item have only selects voiceP
% \end{itemize}


\subsection{Distribution of the -en participle}
A passive participle, but not a past participle, can serve as a modifier of a
noun phrase. see below??


\subsubsection{Noun Modifier}
\begin{description}
\item [?label?] 28a. A book written by John is on the table.
\item [?label?] 28b. *The man written a book just came in.
\item [?label?] 28c. The man who has written a book just came in.
\end{description}


\subsubsection{Absolute Constructions}
What  is absolute in generative terms??
\begin{description}
\item [Absolute Passive] 28a. Written in only three days, this book sold millions of copies.
\item [Absolute supposed to be??] 28b. *Written her dissertation in only three days, Sue took a break.
\item [Absolute Past] 28c. Having written her dissertation in three days, Sue took a break.
\end{description}

\section{\emph{By} as the Head of VoiceP}

\subsection{By seems to select a vP}
``Dummy {\emph by} requires a vP (and not vice versa): if dummy {\emph by} appears, then it
is certain that there is a vP in the structure. On the other hand, if vP appears,
there is no guarantee that there will be a dummy {\emph by} in the structure...This asymmetry in selection suggests that {\emph by}
subcategorizes for a vP, and not the other way around. In other words, the
preposition by must be listed with the subcategorization frame [ \_vP]."

\subsection{Three problems with {\emph by} as a PP in Spec vP}
\begin{enumerate}
\item the strict UTAH problem, 
\item the syntactic distribution
\item and the Case-absorption problem.
\end{enumerate}

Collins addresses all three by postulating that {\emph by} doesn't form a consitutent with the following DP. Instead its a functional head consisting entirely of uninterpretable feature which checks the accusative Case of the DP in Spec,vP, in a way that is very similar to how the prepositional complementizer \emph{for} checks the case of a DP in Spec,IP in phrases like [CP For John to win would be nice].

% In minimalist syntax, $v$ checks accusative Case and assigns the
% external $theta$-role. These two features are distinct, so it is natural to ask whether they can ever be dissociated. Collins suggests
% that it is precisely in the passive that the two features come apart and are projected on two different heads:
% 
% (31) a. active: v assigns external $theta$-role
% v checks accusative Case
% b. passive: v assigns external $theta$-role
% Voice [by] checks accusative Case

\section{Smuggling}

\begin{example}
Smuggling:\\
``Suppose a constituent YP contains XP.
Furthermore, suppose that XP is inaccessible to Z because of the presence of
W (a barrier, phase boundary, or an intervener for the Minimal Link Condition
and/or Relativized Minimality), which blocks a syntactic relation between Z
and XP (e.g., movement, Case checking, agreement, binding). If YP moves to
a position c-commanding W, we say that YP smuggles XP past W. This is
illustrated as follows:
In this example, YP is the smuggler, XP is the smugglee, and W is the
blocker.''
\end{example}

\subsection{Problem with phases and strong heads}
{\bf Problem}
Chomsky (2001a:12, 43, fn. 8; 2001b:25) suggests
that v* (v with full argument structure) is a strong phase head. For Chomsky,
the v found in passives and unaccusatives does not count as a strong phase
head, because it lacks an external argument.

{\bf Solution(?)} After PartP has moved to Spec,VoiceP, PartP is in a sense
dissociated from the external argument that has been left behind in Spec,vP. So
this PartP is like an unaccusative (for which vP does not have an external
argument). Therefore, neither the moved PartP nor an unaccusative vP are
strong phases.


\subsection{Two ways to get what are traditionally a called passives}
\begin{description}
\item [passive suffix] the verb raises to v, which in turn raises to Voice. Verb movement to Voice does not
allow for smuggling to take place. A Voice head must attract V: either V is embedded in PartP. If there is a preposition for the external DP it will probably function as a preposition in other environments or as a case marking postposition.
\item [passive head] a preposition is reanalyzed as a functional head which subcategorizes for a vP. 
\end{description}

\section{The Passive without the By-Phrase (Short Passives)}

Even though the external argument in short passives is not phonetically overt it can bind a reflexive or license a depictive secondary predicate. Thus it is syntactically
present:
\begin{example}
 Such privileges should be kept to oneself.
\end{example}

\subsection{Parallel between by-phrases and infinitivals}
This is similar to the relationship between infinitivals
clauses with an overt lexical subject and infinitival clauses with a PRO subject.
\begin{example}
a. For John to win would be exciting.\\
b. PRO to win would be exciting.
\end{example}

\subsection{Problems with the null Voice\textsuperscript{o}}
The Null case analysis suffers from many difficulties which can be solved if we postulate that Comp, not Infl,
checks null Case.

\begin{tabular}{|l|l|}\hline
Has a Null comp: & ``to win would be exciting" \\
Has no Comp: & ``John seems to be nice"\\\hline
\end{tabular}

``Therefore, in (48b) the minimal Comp checks the null Case of PRO
in Spec,IP under c-command. Extending this analysis to the null
argument in the passive, we can say that null Voice checks the Case of
the PRO found in the short passive. Therefore, the empty category in
the passive in (48b) is simply arbitrary PRO. "

\subsection{Differences between By and for}
\begin{enumerate}
\item by takes a vP complement,
\item for takes an IP complement. 
\item for does not trigger XP (PartP) movement to its specifier.
\end{enumerate}

\section{Remnant Movement and Stranding}
Resltative secondary predicates are much better when they follow the external argument, which suggests they are pied-piped with the PartP.

Consider resultative secondary predicates:
\begin{example}
 a. The table was wiped clean by John.\\
b. ??The table was wiped by John clean.\\
c. The metal was hammered flat by John.\\
d. ??The metal was hammered by John flat.\\
\end{example}

For every possible test, the external argument
c-commands what follows it (including PPs, IPs, and CPs) in the above
structures.

\section{Derived Constituent Structure}
\subsection{Problem: violation of the Minimal Link Condition or Relativized Minimality.}

These data seem to show that the sequence by DP is a constituent and
that, furthermore, it is a PP. But we see the same syntactic constraint with the complementizer for.
\begin{example}
For John to leave would be unfortunate.
\end{example}

\subsection{Conjoined By phrase are elided PartPs}

The underlying structure
must be one where two VoiceP projections are conjoined:
\begin{example}
a. The book was written by John and by Bill.\\
b. The book was written [ConjP [PP by John] and [PP by Bill]].
\end{example}

\subsection{Difference between for and by}
The DP following for cannot be extracted: 

\begin{example}
a. *Who would you prefer for to win?\\
b. Who was the book written by?
\end{example}


if VoiceP undergoes movement, everything
except Spec,vP must be evacuated.

\section{The Binding Theory in the Passive}
"The data involving reflexives and reciprocals in the
passive are murky." 

\section{C-command of By-phrase}

Since there is no principle C
effect in the passive, then it must be the case that reconstruction is not
obligatory.


Because PartP is not a quantificational
expression, reconstruction is not obligatory (and furthermore,
only marginally possible) with participle movement. 

\section{Conclusion}
``What are the parameters in my analysis? They are all of the form X (X a
functional head, composed uniquely of uninterpretable features) exists in L (an
I-language). One such parameter is the existence of the past/passive participle
functional head -EN, which I have argued to be composed of uninterpretable
features. Another is the existence of the VoiceP."
\end{document}