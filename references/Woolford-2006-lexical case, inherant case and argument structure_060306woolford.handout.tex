\documentclass[12pt]{article}
\usepackage{covington}
%\usepackage{fullpage}
\usepackage{tree-dvips}
\usepackage{qtree}
\usepackage{tipa}
\usepackage{2up}
%\usepackage{ulem}

\let\ipa\textipa %to use \ipa rather than \textipa

\author{Ellen Woolford\\1997}
\title {Four-way Case Systems: Ergative, Nominative, Objective and Accusative}
\date{Handout by Gina Cook\\March 07 2006}
\begin{document}
\pagenumbering{arabic}
\maketitle

Three main points using data from Nez Perce\footnote{Nez Perce is a Sahaptian language spoken in the northwestern U.S.} and Kalkatungu:
\begin{enumerate}
\item Ergative is a lexical case, not a structural case.
\item UG allows for two object case positions, \\
-Spec Agr-O ``objective case''\\
-inside VP ``accusative case''
\item Crosslinguistic Generalization: ``in a clause with a lexically cased subject the highest object cannot have structural accusative case.''
\end{enumerate}


We are only focusing on the first point.
\subsection*{Background}



Case theory predicts a lexical or quirky case  (assigned at D-structure) associated with agent theta roles. Woolford claims that ergative is associated with agents as much as dative is with  experiencer/goals.

\begin{example}
Case and agreement in sentence types Nez Perce:\\
\begin{tabular}{|lll|}\hline
INTRANS  & Nom = \ipa{\o} & agree= $\sqrt{}$\\\hline
(DI)TRANS & Nom = \ipa{\o} & agree= $\sqrt{}$\\
& Acc = \ipa{\o} & agree= X\\
& (ACC =\ipa{\o} & agree= X)\\\hline
(DI)TRANS & Erg = m & agree=$\sqrt{}$\\
& Obj = na & agree=$\sqrt{}$\\
& (ACC =\ipa{\o} & agree= X)\\\hline
(DI)TRANS & Erg = m & agree=$\sqrt{}$\\
& Obj = na & agree=$\sqrt{}$\\
& (DAT = na & agree = X)\\\hline
\end{tabular}
\end{example}

\begin{example}
NP positions in the tree:\\
\Tree[.Agr-SP {$NP-Nom$\\$NP-Erg$} [.Agr-S' Agr-S [.AgrO-P $NP-Obj$ [.Agr-O' Agr-O [.VP [.V' V $NP-Acc$\\$NP-Dat$ ] ] ] ] ]  ]
\end{example}

\begin{example}
Translated Tree for where we finished last week?\\
\Tree[.aspP . [.asp' asp\textsuperscript{o} [.$v$P  {$NP-Nom$\\$NP-Erg$} [ $NP-Obj$ [.$v$' $v$\textsuperscript{o} [.VP [.V' V $NP-Acc$\\$NP-Dat$ ] ] ] ] ]  ] ]
\end{example}


\subsection*{Section 1.2 Ergative as a Lexical Case}

``the correlation between ergative Case and
agents is strong enough to justify the view that ergative is the lexical Case
associated with agents.''

\begin{itemize}
\item `Classic' ergative language - marks  only transitive subjects as Erg
\item `Active' ergative language - also marks intransitive agents as Erg
\end{itemize}

``There is a feeling that if the ergative Case were really a lexical Case associated with agents, it would not be limited to transitive clauses...[but this split] is typical of a lexical case''

%compare with  dative subjects

Non-agentive subjects are often not
marked with ergative Case. Instead non-agentive subjects may get dative
Casc (e.g., Hindi) 

When the subject is either dative or ergative, the object becomes
nominative and triggers agreement

\subsubsection*{Syntactic Ergativity, Dixon 1972, 1979}

``A language is
classified as syntactically ergative if it has syntactic rules that refer to or
apply to ergative NPs or absolutive NPs.''

%treating ergative as a lexical case is not only compatible with morphological Ergativity but also opens new doors for syntactic ergativity

NPs with Lexical/Inherent  Case are syntactically Inert is many circumstances. In German only structurally cased objects can  passivize.

Bittner and Hale (1996b) argue that Dyirbal is syntactically ergative because
nominatives move out of VP, while ergatives remain inside the VP

\subsubsection*{Split Ergativity}
\begin{example}
Three types of split:
\begin{enumerate}
\item Case is ergative (assigned based loosely on theta role) while\\
 Agreement is Nom-acc (associated with structural positions)

\item Case is  Erg-Abs while\\	
Case is Nom-Acc
\item Person 3rd Erg-Abs while \\
Person 1st/2nd Nom-Acc
\end{enumerate}
\end{example}

[In Hindi] Some verbs, e.g. `buy', never take an ergative
subject, even in transitive perfective constructions ( Comrie 1984, p. 858).
Such lexical exceptions support the idea that the ability to assign lexical
ergative Case is part of a verb's lexical entry	'


\begin{example}
Counter example, Hindi perfective `buy' does take an Erg subject:
\gll vo to ma-ne kal khridliya
that(masc) emph I(fem)-erg yesterday bought-masc
\glt "Actually I bought \emph{that} yesterday."
\glend
\end{example}

\subsubsection*{Summary}
Nothing is a problem for her idea that ergative is as much a lexical case as dative is, although she provides no rigorous arguments for anything.
\section*{References}
\begin{reflist}
Woolford, Ellen. 1997. ``Four-way Case Systems: Ergative, Nominative, Objective and Accusative.'' NLLT 15, 181-227.
\end{reflist}
\end{document}

