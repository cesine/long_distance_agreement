\documentclass[12pt]{article}
\usepackage{covington}
\usepackage{fullpage}
\usepackage{tree-dvips}
\usepackage{qtree}
\usepackage{tipa}
%\usepackage{ulem}

\let\ipa\textipa %to use \ipa rather than \textipa

\author{Stanislas Dehaene, Ghislaine Dehaene-Lambertz and Laurent Cohen}
\title {Abstract representations of numbers in the
animal and human brain}
\date{February 22 2006}
\begin{document}
\pagenumbering{arabic}
\maketitle
%\section*{Preliminaries}
%I'm not familiar with this sort of literature, or with conceptions of number/quantity beyond what I learned in highschool and in syntax. Please correct me if I phrase things non-conventionally/incorrectly. :)
%\section{Background}

%Skeptic of non-human number processing due to an incorrectly analyzed experiment that claimed a horse (clever Hans)  had the mastery of symbolic calculation. 

\section{Criteria for an abstract representation of numbers}

\subsection*{Arguments for a network/module/faculty}

\begin{itemize}
\item spontaneous emergence at a young age in all healthy members of a species
\item presence across species (ie, humans and animals)
\item lesion \& brain-imaging studies indicate a specific neural substrate
\end{itemize}

\section{Number processing in infants}

\subsection*{Sample Experiment - Compairing number/quantity}
Infants were habituated to a slide with a fixed number of dots (eg, two)
At habituation, the presentation of
slides with a different number of dots
yielded significantly longer looking
times, indicating dishabituation and therefore discrimination
between two and three.

\subsection*{Sample Experiment - Adding/Subtracting}
Show a toy hidden behind the screen, add one or take one away. Change the identity of the objects (so that its not their identity that matters) and remove screen. Children look significantly longer at incorrect additions and subtractions.

\section{Number processing in animals}
The above experiments were also performed on animals with similar results. In addition, cross-modality experiments were performed on animals with similar results. This indicates abstraction from visual or auditory representations to abstract representations of number.

\subsection*{Sample Cross-modality Experiment - Compairing number/quantity}
Rats trained initially on distinct
auditory and visual discrimination tasks were shown
later to generalize to novel sequences in which auditory
and visual stimuli were mixed.

\subsection*{Manipulation of symbolic representations of number}
Monkeys and chimpanzees have been
taught to recognize the Arabic digits 1 - 9 and to use them
appropriately to refer to sets of objects.\\
-Caveat: ``such experiments
cannot be taken to indicate that exact symbolic
or ‘linguistic’ number processing is within the normal
behavioral repertoire of animals. However, they do
indicate that abstract, presumably non-symbolic representations
of number are available to animals and
can, under exceptional circumstances, be mapped on
to arbitrary behaviors that can then serve as numerical
‘symbols’ ''

\section{Parallels between animal and human
representations of number}

Humans and animals show two effects:

\subsection*{Numerical distance effect}
The ability to discriminate
between two numbers improves as the numerical distance between them increases.

\subsection*{The number size effect}
For equal numerical distance, discrimination of
two numbers worsens as their numerical size increases.\\~\\
-Caveat: ``Comparison times and error rates are a continuous,
convex upward function of distance, similar to psychophysical
comparison curves.''\\
$\rightarrow$ 
Question for Discussion: because animals and humans share these effects, and these effects are also seen in other comparisons, are these effects due to number, comparing, or are the effects seen in other comparisons derived from using the number ``network''?

\section{Deficits of semantic number processing in
brain-lesioned patients}

\subsection*{Lesions in Areas of the Brain\footnote{Needs fleshing out}}
\begin{itemize}
\item inferior intra-parietal area of left hemisphere lesions\\
-might result difficulties in comprehending, producing and calculating with numbers
\item inferior intra-parietal area of right hemisphere lesions
\item parietal acalculia lesions\\
-might result in ``disorganization of an abstract
semantic representation of numerical quantities rather than of calculation processes per se''
\item dominant-hemisphere inferior parietal
lesions and Gerstmann’s syndrome.\\
-might be specific impairments in subtraction and number
bisection, suggesting disturbance to the central
representation of quantities.
\end{itemize}

\section{Brain-imaging studies of number processing}

\subsection*{Processing in Areas of the Brain\footnote{Needs fleshing out}}
\begin{itemize}
\item left parietal area \\
-multiplication yielded greater activity
\item left inferior parietal area
\item right parietal area
\item right inferior parietal area \\
-Relative to letter reading, digit comparison yielded greater
activity
\end{itemize}


%\subsection*{Brain - Compairing number/quantity}
\section{``Take home message''}
We have good understanding of 2 sorts of number/quantity related effects exhibited by animals, infants and adult humans. We have some evidence for where number/quantity processes take place in the brain (parietal area). Further research/models are needed to relate these effects to processes in the brain. Language impairment can confound the investigation number/quantity impairment.

%Language is involved in some number/quantity processing, such as manipulation of symbolic representations of number (Arabic numerals) and rote learning (multiplication \& addition tables). 

%\subsection*{Brain - exact arithmetic}


\section{Some missing links and pointers to further
research}

\subsection*{Lacking}
\begin{itemize}
\item directly parallel studies in animals and humans
\item studies of higher numbers than 6 in infants
\item  involvement of the intraparietal
cortex in infants and animals remains speculative.
Non-invasive brain-imaging techniques applicable to
infants have not been applied to number processing.
\item lesion and electrophysiological studies of
number in animals.
\end{itemize}


\end{document}