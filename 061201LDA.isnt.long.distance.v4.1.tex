\documentclass{article}
\usepackage{covington}
\usepackage{fullpage}
%\usepackage{tree-dvips}
\usepackage{qtreegina}
\usepackage{xylinggina}
%\usepackage{xyling}
\usepackage{tipa}
%\usepackage{ulem}
\usepackage{natbib} %citation style

\let\ipa\textipa %to use \ipa rather than \textipa

\author{Gina}
\title {Hindi-Urdu Long Distance Agreement isn't Long Distance}
\date{November 2006}
\begin{document}
\bibliographystyle{linquiry2}
\pagenumbering{arabic}
\maketitle
\tableofcontents

\section{Introduction}

Long Distance Agreement is when the matrix verb agrees with an embedded object. It is parasitic on agreement with the embedded verb, typically a participle (in Icelandic). In Hindi-Urdu this embedded verb is in the -na form which has traditionally been considered an infinitive (Mahajan 1990, Bhatt 2005, Hiraiwa 2006). In this paper I show that it is a nominalizer. I argue that Long Distance Agreement in Hindi-Urdu can be explained by two typical agreement relations, and thus should not be used in arguments for changing the mechanism of Agree (Bhatt 2005) or for its original argument distinguishing A-movement scrambling and A-bar movement scrambling (Mahajan 1990). 


\subsection{What is Long Distance Agreement in Hindi-Urdu}
Verbal affixes, and auxiliaries agree in gender or person with the highest unmarked argument.


\begin{example}Verbal affixes which show gender not person\\
\begin{tabular}{|l||c|ll|cc|}\hline
Gender&\sc{NonFin/(Present?)}  &\sc{past}  &\sc{future} & \sc{imperfect} & \sc{perfect} \\\hline\hline
Masc &~-na &  =t\super ha & =ga & -ta & -ya  \\
Fem &-ni & =t\super hi & =gi & -ti & -yi  \\
Plural &~-ne &  =t\super he & =ge & -te & -ye  \\\hline
Obj of Postposition/Complementizer & ~-ne &  =t\super he? & =ge? & -te? & -ye?  \\\hline
\sc{Subjuctive} & ~-ne? &  =t\super he? & =ge? & -te? & -ye?  \\\hline
\end{tabular}
\end{example}

\begin{example}Verbal affixes which show person not gender\\
\begin{tabular}{|l||c|lll}\hline
Person &\sc{present/imperative(Finite?)/(copula?)}  \\\hline\hline
%Ben-give & Ben-take 
1sg & -\~{u} \\
2sg & -o \\
3sg & -\ipa{e} \\\hline
 &\\
pl &  -\~{e} \\
 & \\\hline
\end{tabular}
\end{example}

Subjects can either marked with -\o, -ne or -ko. Subjects are marked ergative \textit{-ne} in the perfective. Subjects of experiencer verbs are marked with  \textit{-ko}. Animate objects are always marked with \textit{-ko}. Animates and inanimates which are specific are also marked with \textit{-ko}. Some verbs always mark their object with \textit{-ko} such as \textit{ma\:r} `hit' where the patient is always marked with -ko, but an optional instrument is unmarked. This causes the need to be clear about the theta role of the ``object." The chart below shows examples of verbs, varied by their arguent's theta roles.
 

\begin{example}
\begin{tabular}{|l|lll|ll|l|}\hline
&Subject &  DObject & IDObject  & Verb Example& & Agreement \\\hline\hline
Non-perfect&Agent-\o &  Theme-\o/-ko && k\super ha- & `eat'  & Subject \\
&Agent-\o &  Instr-\o/-ko? & Patient-ko  & \ipa{ma\:r}- & `hit' & Subject\\
&Agent-\o &  Instr-\o/-ko? & Loc-se  & \ipa{nI".kal-} & `take.out' & Subject\\
&Agent-\o&Theme-\o?/ko&& \ipa{"p2.s@nd kar-} & `like do' & Subject\\\hline
&Exp-\o & Theme-\o/-ko? & & \ipa{s@m.d\super Z-} & `understand' & Subject\\
&Exp-\o &&Source-se& \ipa{da\:rt}- & `fear` & Subject\\
&Exp-ko & Theme-\o/-ko && \ipa{p@t-2} & `know' & DO?/Default\\
&Exp-ko&&Instr-sat&tang a- & `annoy come' 	& DO?/Default\\
&Exp-ko&Theme-\o/-ko?&& \ipa{p@".s2nd h-} &  `like be' & DO?/Default \\\hline\hline

Perfective&Agent-ne &  Theme-\o/-ko && k\super ha- & `eat'  & DO/Default? \\
&Agent-ne &  Instr-\o/-ko? & Patient-ko  & \ipa{ma\:r}- & `hit' & DO/Default?\\
&Agent-ne&Theme-\o?/ko&& \ipa{p@s2nd kar-} & `like do' & DO/Default?\\\hline

(not possible?)&? & Theme-\o/-ko? & & \ipa{s@md\super Z-} & `understand' & ?\\
&? &&Source-se& \ipa{da\:rt}- & `fear` & ?\\
&? & Theme-\o/-ko && \ipa{p@t-2} & `know' & ?\\
&?&&Instr-sat&tang a- & `annoy come' 	& ?\\
&?&Theme-\o/-ko?&& \ipa{p@s2nd (cop)} &  `like be' & ?\\\hline
\end{tabular}
\end{example}

Below is a sample sentence.

\begin{example}Subject agreement in Non-perfect (Mahajan 1990:3)
\glll siitaa kelaa khaatii thii
sita-{\o}={\o} kel-a={\o}   k\super ha-t-i t\super h-i 
Sita.F={\o} banana-M={\o} eat-\sc{Impf}-F~ =\sc{Past}-F
\glt `Sita used to eat banana(s).'
\glend
\end{example}

% \begin{example}Subject agreement in Non-perfect (Mahajan 1990:3)
% \glll raam roTii khaataa thaa
% raam-{\o}={\o} \ipa{ro\:t-i=\o}   k\super ha-t-a t\super h-a 
% Ram.M={\o} bread-F={\o} eat-\sc{Impf}-M~ =\sc{Past}-M
% \glt `Ram used to eat bread.'
% \glend
% \end{example}

\begin{example}Object agreement in Perfect (Mahajan 1990:5)
\glll raam-ne roTii khaayii thii
raam-{\o}=ne \ipa{ro\:t-i=\o}   k\super ha-y-i  t\super h-i 
Ram.M=\sc{erg} bread-F={\o} eat-\sc{perf}-F =\sc{Past}-F
\glt `Ram had eaten bread.'
\glend
\end{example}


There is a puzzling agreement pattern refered to as Long Distance Agrement. When the matrix subject is marked with -ne, both the embedded and matrix verb agree with the embedded object. 

\begin{example}Long Distance Agreement in Hindi-Urdu (Mahajan 1990:4)
\glll raam~ne rotii khaanii caahii
ram=ne \ipa{ro\:t-i=\o} k\super ha-n-i \ipa{t\super Sah}-\o-i
Ram.M=\sc{erg} bread-F eat-\sc{NonFin}-F want-\sc{past}-F
\glt `Ram wanted to eat bread.'
\glend
\end{example}



\begin{example}Bhatt: LDA and object agreement are the same, they differ only in the distance\\
\Treek{1}{ 
Object\\
& TP\B{dl}\B{dr}\\
Subj-ne && T'\B{dl}\B{dr} \\
& aspP\B{dl}\B{dr}&&T^o\\
Subj-ne && asp'\B{dl}\B{dr} \\
& vP\B{dl}\B{dr}&&asp^o\B{d}\\
Subject && v'\B{dl}\B{dr} &Perfect\\
& VP\B{dl}\B{dr}&&v^o\\
DObj && V'\B{dl}\B{dr} \\
& (IDObj)\footnote{Hale \& Keyser 1993, Chomsky 1995} &&V^o\\
}
\Treek{1}{ 
LDA\\
& TP\B{dl}\B{dr}\\
Subject && T'\B{dl}\B{dr} \\
& aspP\B{dl}\B{dr}&&T^o\\
DP && asp'\B{dl}\B{dr} \\
& vP\B{dl}\B{dr}&&asp^o\\
Subject && v'\B{dl}\B{dr} \\
& VP\B{dl}\B{dr}&&v^o\\
DP && V'\B{dl}\B{dr} \\
& TP\B{dl}\B{dr}&&V^o\\
(PRO) && T'\B{dl}\B{dr} \\
& aspP\B{dl}\B{dr}&&T^o\\
DP && asp'\B{dl}\B{dr} \\
& vP\B{dl}\B{dr}&&asp^o\\
(PRO) && v'\B{dl}\B{dr} \\
& VP\B{dl}\B{dr}&&^o\\
(nothing?) && V'\B{dl}\B{dr} \\
& Object &&V^o\\
}
\Treek{1}{ 
& TP\B{dl}\B{dr}\\
DP && '\B{dl}\B{dr} \\
& P\B{dl}\B{dr}&&^o\\
DP && '\B{dl}\B{dr} \\
& P&&^o\\
}
\end{example}


\section{Why is Long Distance Agreement Interesting?}

\subsection{LDA is a challenge to types of Movement or Case or Specificity}
Long Distance Agreement was originally used as an argument distinguishing two types of scrambling, A scrambling and A bar scrambling. This had ramifications on the understanding of specificty and structural case marking, namely (as I understand it currently) that Hindi was the opposite of other languages, structurally case marked objects (objects which didnt agree with the V) were non-specific and didnt move further. While objects which agreed with the verb were structurally case marked by AgrO and had undergone movement and so were specific. 

\subsection{LDA is a challenge for Bounded A movement}
Mahajan argues that the object in LDA moves from an embedded clause to a matrix clause. He says it similar to clitics in Romance. Boecks says  it is too, and whatever accounts for this will account for that. Boecks says its a movement that doesn't cross a phase boundary, the matrix verb is a restructuring verb. I have to show its not a restructureing verb, but is the matrix verb

\subsection{LDA is a challenge for Agree}
\subsection{Counter Cyclic}

\section{Why it might not be ``Long-Distance" at all}

\subsection{The complement of the matrix verb is a Nominalized Clause}

\subsubsection{Its subject is genitive}

\subsubsection{It can be case marked}

\subsubsection{It has the distributions of DPs}

\section{LDA is the result of 2 agreement relations}

\subsection{Normal agreement in the embedded predicate}

\subsection{Normal agreement in between the matrix verb and the nominalized clause, NOT the embedded object}

\section{LDA isn't Restructureing}
\subsection{Get some tests from Bobalijk and wurmbrand}

\section{When does LDA happen?}

\subsection{Test with informant: does the LDA only happen when the object is specific? (Mahajan 1990, Bhatt 2005)}

%\section{Further issues: How does agreement work in hindi outside of LDA?}


\appendix

\section{Verb Final Structures}

\begin{example} Cinque’s (1999?) functional hierarchy \\
$MoodPspeech act > MoodPevaluative > MoodPevidential > ModPepistemic > TP(Past) > TP(Future) > MoodPirrealis > ModPalethic > AspPhabitual > AspPrepetitive(I) > AspPfrequentative(I) > ModPvolitional > AspPcelerative(I) > TP(Anterior) > AspPterminative > AspPcontinuative > AspPretrospective > AspPproximative > AspPdurative > AspPgeneric/progressive > AspPprospective > ModPobligation > ModPpermission/ability > AspPCompletive > VoiceP > AspPcelerative(II) > AspPrepetitive(II) > AspPfrequentative(II)$
\end{example}

\begin{example}Causative/incoative structures (Hornstein et all 2006: understanding minimalism Capter 3:97, X is for any lexical category that can go with a light verb, dont know if they meant all these X, in these examples the roots are verbs, try with nouns) external arguments of ditransitive, simple transitive and unergative are in spec vP, internal arguments are inside the shell of hte contentfull verb.\\
Is the spec and comp of the give in ditrans right??\\
\Treek{1}{ 
Di &trans\\
& vP\B{dl}\B{dr}\\
John && v'\B{dl}\B{dr} \\
& XP\B{dl}\B{dr}&&v^o\B{d}\\
gift && X'\B{dl}\B{dr} & DOES\Below{+EPP} \\
& to Mary&&X^o\B{d}\\
& &&gave\Below{+EPP}
}
\Treek{1}{ 
Causeative\\
& vP\B{dl}\B{dr}\\
the~army && v'\B{dl}\B{dr} \\
& XP\B{dl}\B{dr}&&v^o\B{d}\\
 the~ship&&X^o\B{d} & DOES\Below{+EPP}\\
 &&sank\Below{-EPP}
}
\Treek{1}{ 
Incoative &and& Unacc?\\
& XP\B{dl}\B{dr}\\
 the~ship&&X^o\B{d}\\
 &&sank\Below{-EPP}
}
\end{example} 	

\begin{example} Unerg (laugh, cough, fall) and Light + noun (give a sigh, take a turn, do lunch) paralleles (hornstein et all 2006 understading minimalism chapter 3:101)\\
\Treek{1}{ 
Unerg\\
& vP\B{dl}\B{dr}\\
John && v'\B{dl}\B{dr} \\
& XP\B{d}&&v^o\B{d}\\
&X^o\B{d} && DOES\Below{+EPP}\\
&sighed\Below{-EPP}
}
\Treek{1}{ 
light verb\\
& vP\B{dl}\B{dr}\\
John && v'\B{dl}\B{dr} \\
& XP\B{d}&&v^o\B{d}\\
&X^o\B{d} && give\Below{+EPP}\\
&a~sigh\Below{-EPP}
}
\Treek{1}{ 
??\\
& vP\B{dl}\B{dr}\\
John && v'\B{dl}\B{dr} \\
& XP\B{dl}\B{dr}&&v^o\B{d}\\
a~sigh && X^o\B{d} & DOES\Below{+EPP} \\
 && give\Below{-EPP}\\
}
\end{example}


\begin{example}Complements vs Adjuncts (notes from grohmann ch3)he has a lot more ideas in there...htis is just be basics\\
\Treek{1}{ 
&Xbar\\
& XP\B{dl}\B{dr}\\
Spec && X'\B{dl}\B{dr} \\
& X'\B{dl}\B{dr}&&(Adj)\\
(Adj) && X'\B{dl}\B{dr} \\
& Comp &&X^o\\
}
\Treek{1}{ 
&LCA\\
& XP\B{dl}\B{dr}\\
(Adj) && XP\B{dl}\B{dr} \\
&(Adj)&& XP\B{dl}\B{dr}\\
 && X^o && Comp
}
\Treek{1}{ 
&Groh&mann\\
&& XP\B{dl}\B{dr} \\
& XP\B{dl}\B{dr}&&(Adj)\\
(Adj) && XP\B{dl}\B{dr} \\
&Spec&& X'\B{dl}\B{dr}\\
 && X^o && Comp
}
\end{example}
\begin{example}Types of restruturing, raising, ECM,  causatives, control (grohnman ch 3 p 94)
\end{example}

\subsection{Restructuring, Small complements}
``Whenever a language exhibits restructuring
phenomena, that language has functional restructuring" (Wurmbrand 2004:1012) does hindi?

Types of restruturing, lexical vs functional (wurmbrand 2004 two types of restructuring lexical vs functional)\\
``Restructuring constructions are infinitival constructions which are characterized by the lack of clause-boundedness effects (in languages in which infinitives otherwise show clausal behavior).''
-clitic climbing
-long distance scrambling
-long distance A movement
-long distance agreement
\begin{example}
%\QTree[.FP_n . [.F' F^o\\functional\\RV [.vP SUBJ [.v' v^o [.VP DP [.V' V^o\\main~verb\\(infinitive) [.DP ] ] ] ] ] ] ]
\Treek{1}{ 
Functional\Below{Restructuring}& & (Wurmbrand & 2004ex16)\\
& AuxP\B{dl}\B{dr}\\
 && Aux'\B{dl}\B{dr} \\
& (ModP)\B{dl}\B{dr}&&Aux^o\Below{{\bf functional}}\Below{RV}\\
 && (Mod')\B{dl}\B{dr} \\
& vP\B{dl}\B{dr}&&(Mod^o)\Below{(functional}\Below{RV)}\\
SUBJ-nom && v'\B{dl}\B{dr} \\
& VP\B{dl}\B{dr}&&v^o\Below{(functional}\Below{RV)}\\
(OBJ)\Below{which?} && V'\B{dl}\B{dr} \\
& (OBJ)\Below{which?}&&V^o\Below{main verb}\Below{(infinitive)}\\
}
\Treek{1}{
Lexical\Below{Restructuring} && (Wurmbrand & 2004ex2b)\\
& \ldots\B{dl}\B{dr}\\
 && \ldots'\B{dl}\B{dr} \\
& vP\B{dl}\B{dr}&&\ldots^o\\
SUBJ-nom && v'\B{dl}\B{dr} \\
& VP\B{dl}\B{dr}&&v^o\\
IO-dat && V'\B{dl}\B{dr} \\
& VP\B{dl}\B{dr}&&V^o\Below{{\bf lexical}}\Below{RV}\\
(PRO?) && V'\B{dl}\B{dr} \\
& (DP)\Below{which?}&&V^o\Below{infinitive}\\\
}\\
Functional (always) Restructuring Predicates: must, seem\\
Lexical (potential) Restructuring Predicates: try, dare, forget, intend, forbid, recommend, allow, want(Italian)\\
Lexical (potential) Restructuring Unaccusative, assign a internal dative argument Predicates: manage, allow, recommend, fail \\
~~(unaccusatives, select dative experiences which cant be controlled so these cant be embedded  infinitves)
\end{example}

\begin{example}Add the LF in Wurmbrand 2004ex27 to show the phases
\end{example}

\begin{example}Properties of Functional Restructuring verbs \\
Double check the table wurmbrand 2004:1012)\\
\begin{tabular}{|l|c|c|c|c|}\hline
& functional morph & functional restruct & lexical restruct & lexical  verbs\\\hline\hline
verbs & & & & \\
thematic domain & Above & Above & Below & Below\\
& (Wurmbrand 2004)& (Wurmbrand 2004) & (Wurmbrand 2004)& (Wurmbrand 2004)\\\hline
complements & vP,MP & vP,MP & VP(not vP?) & VP,TP,CP,AgrP\\
\\\hline
assign $\theta$ roles & NO & NO & YES & YES \\
& (Cinque 2001)& (Cinque 2001) & (Wurmbrand 2004)& (Wurmbrand 2004)\\\hline
external arguments & NO & NO & YES & YES \\
& (Cinque 2001)& (Cinque 2001) & (Wurmbrand 2004)& (Wurmbrand 2004)\\\hline
internal arguments& NO & NO & YES & YES \\
& (Cinque 2001)& (Cinque 2001) & (Wurmbrand 2004)& (Wurmbrand 2004)\\\hline
raising & YES & YES & NO & NO\\
& (Cinque 2001)& (Cinque 2001) & (Wurmbrand 2004)& (Wurmbrand 2004)\\\hline
weather-it & YES & YES & NO & NO\\
Inanimate arguments & YES & YES & NO & NO\\\hline
allow passivizing & NO & NO & YES & YES \\
& (Cinque 2001)& (Cinque 2001) & (Wurmbrand 2004)& (Wurmbrand 2004)\\\hline
clitic climbing & YES & YES & YES & NO\\\hline
OBJ scrambling to matrix & YES & YES & YES & NO \\\hline
Case of OBJ & NOM by matrix v?& NOM by matrix v?& NOM by matrix v?& ACC by embed v?\\\hline
long A-move & YES & YES & YES & UNACC-YES?\\\hline
long passive & - & YES & YES & -\\\hline
Low Reconstruction & YES & YES & NO & NA?\\\hline
Mono-clausal & mono & mono & bi & bi\\\hline
\# of Adverbs & ONE & ONE & TWO & TWO \\\hline
Scope modal/obj & BOTh & BOTH &$ obj<modal$ &$ modal<obj$?\\\hline
variying complements & NO & NO & YES & YES \\
& (Cinque 2001)& (Cinque 2001) & (Wurmbrand 2004)& (Wurmbrand 2004)\\\hline
rigid ordering & YES & YES & NO & NO\\
& (Cinque 2001)& (Cinque 2001) & (Wurmbrand 2004)& (Wurmbrand 2004)\\\hline
co-ocurrance restr &  YES & YES & NO & NO\\
& (Cinque 2001)& (Cinque 2001) & (Wurmbrand 2004)& (Wurmbrand 2004)\\\hline
RC Pied Piping & NO & NO & NO & YES \\\hline	
extrapose infinitive? & NO? & NO & YES? &NO \\\hline
\end{tabular}\\
long passive is formed by passivizing the lexical (main) predicate. its oblig for the embedded object to raise and be nominative
\end{example}

\begin{example}
Types of Sentential complements\\
CP, IP, small clause, noun phrases
\end{example}

\begin{example}Weak and strong Crossover
\end{example}

\begin{example}Islands
\end{example}

\begin{example}Tough movement and opperators (rel clause, comparatives) (Hicks 2003 reviews GP and minimalist approaches, referneced in Hornstein et al, 2006 understanding minimalism ch 6:59)
\end{example}

\begin{example}DP structure
nouns, gerunds
\end{example}

\begin{example}Clasifiers vs no clasiviers vs partitives
\end{example}

\section*{References}

\begin{reflist}

Baker, M 1985. Incorporation: A thoery of Grammatical Funciton Changing. Doctoral dissertation. MIT.

Bains, G.S. 1987. Complex Structures in Hindi-Urdu: Explorations in Government and Binding Theory. Doctoral dissertation. NYU.

Belletti, A. 1988. Unaccusatives as Case Assigners. \textit{Linguistic Inquiry}. 19.1.

Chomsky, N. 1986a. \textit{Barriers}. MIT Press, Cambridge.

Chomsky, N. 1986b. \textit{Knowledge of Language}. New York: Praeger.

Comrie, B. 1984. Reflections on Verb Agreement in Hindi. \textit{Linguistics}

Saksena, A. 1981. 

\end{reflist}

\bibliography{rht,kbj}
\end{document}